\section{Анализ предметной области}
\label{sec:domain}

В данном разделе будет произведён обзор предметной области задачи, решаемой в рамках дипломного проекта.

\subsection{Теоретические основы распознавания номеров}
\label{sub:domain:theory_basics}
Задача распознавания автомобильных номерных знаков относится к тами научным областям как теория распознавания образов и обработка изображений.
Обработка изображений — любая форма обработки информации, для которой входные данные представлены изображением, например, фотографиями или видеокадрами. Обработка изображений может осуществляться как для получения изображения на выходе (например, подготовка к полиграфическому тиражированию, к телетрансляции и т. д.), так и для получения другой информации (например, распознание текста, подсчёт числа и типа клеток в поле микроскопа и т. д.). Кроме статичных двухмерных изображений, обрабатывать требуется также изображения, изменяющиеся со временем, например, видео.~\cite{image_precessing}
Всю область можно охарактеризовать как молодую, разнообразную и динамично развивающуюся. Интенсивное её изучение началось лишь в конце 1970-х гг., когда компьютеры смогли управлять обработкой больших наборов данных, какими являются изображения. И сейчас нет стандартной формулировки этой области, а многие методы и приложения всё ещё находятся на стадии фундаментальных исследований. В последнее время наблюдается повышение активности изучения области, ввиду всё большего применения её методов в коммерческих продуктах.
Теория распознавания образа — раздел информатики и смежных дисциплин, развивающий основы и методы классификации и идентификации предметов, явлений, процессов, сигналов, ситуаций и т. п. объектов, которые характеризуются конечным набором некоторых свойств и признаков.
Для оптического распознавания образов можно применить метод перебора вида объекта под различными углами, масштабами, смещениями и т. д. Для букв нужно перебирать шрифт, свойства шрифта и т. д.
Второй подход — найти контур объекта и исследовать его свойства (связность, наличие углов и т. д.)
Ещё один подход — использовать искусственные нейронные сети. Этот метод требует либо большого количества примеров задачи распознавания (с правильными ответами), либо специальной структуры нейронной сети, учитывающей специфику данной задачи.

В целом задача распознавания автомобильного номера сводится к следующим: 
\begin{itemize}
  \item Обработка изображения, будет рассматриваться в главе~\ref{sub:domain:image_processing}
  \item Поиск номерного знака
  \item Поиск отдельных символов на рамке с номером
  \item Масштабирование 
  \item Распознавание символа
\end{itemize}

\subsection{Фильтрация изображений}
\label{sub:domain:image_processing}
Под обработкой  изображений понимают семейство  методов  и  задач,  где  входной  и  выходной информацией  являются  изображения~\cite{misoi_clides}.  

