\section{Анализ предметной области}
\label{sec:domain}

В данном разделе будет произведён обзор предметной области задачи, решаемой в рамках дипломного проекта.

\subsection{Теоретические основы распознавания номеров}
\label{sub:domain:theory_basics}
Задача распознавания автомобильных номерных знаков относится к тами научным областям как теория распознавания образов и обработка изображений.
Обработка изображений — любая форма обработки информации, для которой входные данные представлены изображением, например, фотографиями или видеокадрами. Обработка изображений может осуществляться как для получения изображения на выходе (например, подготовка к полиграфическому тиражированию, к телетрансляции и т. д.), так и для получения другой информации (например, распознание текста, подсчёт числа и типа клеток в поле микроскопа и т. д.). Кроме статичных двухмерных изображений, обрабатывать требуется также изображения, изменяющиеся со временем, например, видео.~\cite{image_precessing}

Всю область можно охарактеризовать как молодую, разнообразную и динамично развивающуюся. Интенсивное её изучение началось лишь в конце 1970-х гг., когда компьютеры смогли управлять обработкой больших наборов данных, какими являются изображения. И сейчас нет стандартной формулировки этой области, а многие методы и приложения всё ещё находятся на стадии фундаментальных исследований. В последнее время наблюдается повышение активности изучения области, ввиду всё большего применения её методов в коммерческих продуктах.

Теория распознавания образа — раздел информатики и смежных дисциплин, развивающий основы и методы классификации и идентификации предметов, явлений, процессов, сигналов, ситуаций и т. п. объектов, которые характеризуются конечным набором некоторых свойств и признаков.
Для оптического распознавания образов можно применить метод перебора вида объекта под различными углами, масштабами, смещениями и т. д. Для букв нужно перебирать шрифт, свойства шрифта и т. д.
Второй подход — найти контур объекта и исследовать его свойства (связность, наличие углов и т. д.)
Ещё один подход — использовать искусственные нейронные сети. Этот метод требует либо большого количества примеров задачи распознавания (с правильными ответами), либо специальной структуры нейронной сети, учитывающей специфику данной задачи.

В целом задача распознавания автомобильного номера сводится к следующим: 
\begin{itemize}
  \item Обработка изображений
  \item Поиск номерного знака
  \item Поиск отдельных символов на рамке с номером
  \item Распознавание символа
\end{itemize}

\subsection{Обработка изображений}
\label{sub:domain:image_processing}
Под обработкой  изображений понимают семейство  методов  и  задач,  где  входной  и  выходной информацией  являются  изображения~\cite{misoi_clides}. Обработка изображений обычно преследует одну из следующих целей:
\begin{itemize}
  \item Улучшение качества изображения для восприятия человеком, т.е. сделать изображение лучше с субъективной точки зрения человека
  \item Улучшение изображение для восприятия компьютером, т.е. изменить изображение для упрощения последующего распознавания
\end{itemize}
Типичные задачи для обработки изображений это корректировка яркости, цветов, освещения и устранение шумов. Решение этих задач полезно в контексте подготовки изображения как для человека так и для компьютера. 

Многие алгоритмы распознавания изображение показывают хорошие результаты при правильной пред-обработке. Например при нахождении автомобильной рамки мы будет пользоваться алгоритмом \segmentation{} перед использование которого требуется бинаризировать изображение. Или например пред распознаванием символа требуется произвести масштабирование.

\subsubsection{Бинаризация}
\label{sub:domain:image_processing:binary}
Бинаризация изображений - перевод полноцветного или в градациях серого изображения в монохромное, где присутствуют только два типа пикселей (темные и светлые).\cite{binary_image}
Алгоритм бинаризации несложен. Сперва требуется перевести изображение в оттенки серого использую формулу $ Y = 0.2126R + 0.7152G + 0.072B $, где Y значение градации серого, а R G и B значение каналов исходных цветов. Далее нужно сравнить значение каждого пикселя с неким пороговым значение и если значение превышает порог сделать пиксель темным и светлым если наоборот. 
Существуют различные подходы к выбору порогов, которые условно можно разделить на 2 группы:
\begin{itemize}
  \item Пороговые, которые ищут пороговое значение для изображение целиком
  \item Адаптивные, которые выбирают отдельный порог для каждого участка изображения, обычно используются при неравномерном освещении.
\end{itemize}

\subsubsection{Масштабирование}
\label{sub:domain:image_processing:sizing}

Тут будет инфа про масштабирование

\subsection{Поиск номерного знака}
\label{sub:domain:search}
Задача поиска номерного знака имеет множество решений. Рассмотрим несколько из них.
\subsubsection{Поиск номерного знака с использованием преобразование Хафа}
\label{sub:domain:search:hought}
Преобразование Хафа - алгоритм, численный метод, применяемый для извлечения элементов из изображения~\cite{hough_transform}. Используется в анализе изображений, цифровой обработке изображений и компьютерном зрении. Предназначен для поиска объектов, принадлежащих определённому классу фигур, с использованием процедуры голосования. Процедура голосования применяется к пространству параметров, из которого и получаются объекты определённого класса фигур по локальному максимуму в так называемом накопительном пространстве (accumulator space), которое строится при вычислении трансформации Хафа. 
