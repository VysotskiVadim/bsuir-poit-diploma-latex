\section{Характеристика места практики}
\label{sec:practice:itechart_characteristics}

Компания ИООО~\company{} занимается предоставлением услуг по разработке и тестированию программного обеспечения, поддержке, интеграции и анализа эффективности корпоративных решений для многих европейских и североамериканских компаний.
Основанная в 1993 году, компания EPAM Systems, Inc. (NYSE: EPAM) является признанным лидером в области разработки программного продукта независимыми исследовательскими агентствами. Центральный офис находится в Соединенных Штатах, EPAM обслуживает клиентов в 19 странах по всей Северной Америке, Европе, Азии и Австралии.
EPAM занял 6 место в 2013 году среди 25-и самых быстрорастущих компаний и 3 место в 2014 году среди лучший малых американских компаний по версии журнала Forbes.

Компания предоставляет следующие услуги своим заказчикам~
\cite{epam_2016}:
\begin{itemize}
  \item разработка программного обеспечения;
  Разработка программных продуктов – одна из основных специализаций \company{} с момента основания в 1993 году. За годы успешной реализации проектов для всемирно известных поставщиков ПО, среди которых Oracle, Microsoft, Novell и SAP, \company{} приобрел уникальный опыт в области разработки программного обеспечения, проектирования целевой архитектуры и прототипирования ИТ- решений.
  Около 80 компаний-разработчиков информационных решений по всему миру доверяют \company{} разработку своего программного обеспечения, получая взамен качественный и надежный программный продукт.
  \company{} осуществляет полный цикл разработки программного продукта – с нуля до момента его выхода на рынок, обеспечиваем его дальнейшую поддержку и версионность. Работа над проектом ведется согласно четко выстроенной методологии (Agile/Scrum, RUP, CI (Continuous Integration) и др.), которая может быть гибко адаптирована под процессы конкретного заказчика, включая организацию удаленной работы специалистов. Компания располагает также инструментами собственной разработки для управления проектами и персоналом в территориально распределенной среде.

  \item разработка бизнес-приложений;
  Бизнес стремительно меняется, и компании предпринимают максимум усилий, чтобы реагировать на потребности рынка с опережением. Зачастую это связано с автоматизацией ключевых бизнес-процессов и разработкой приложений, предназначенных для решения специфических для данной отрасли задач. \company{}, успешно реализуя проекты для клиентов со всего мира, приобрела уникальный опыт в области разработки бизнес-приложений для компаний из различных отраслей экономики:
  \begin{itemize}
    \item банков, финансовых и инвестиционных организаций;
    \item страховых компаний;
    \item поставщиков программного обеспечения;
    \item компаний розничной торговли;
    \item телекоммуникационных провайдеров;
    \item нефтегазовых и энергетических компаний;
    \item информационного и медиа-бизнеса;
    \item индустрии путешествий;
    \item автобизнеса;
    \item государственного сектора и др.
        На счету EPAM Systems более 2000 успешно реализованных проектов, из которых 90\% было выполнено в утвержденные сроки и в рамках запланированного бюджета.
  \end{itemize}

  Для повышения эффективности разработки бизнес-приложений в EPAM были созданы выделенные центры компетенции. Здесь происходит накопление и тиражирование отраслевого и технологического опыта компании, постоянный обмен знаниями между участниками проектных команд. Это позволяет нам гарантировать высокий уровень качества разработанных ИТ-решений.
  Разработка бизнес-приложений может осуществляться и по модели выделенного центра разработки. В этом случае \company{} формирует команду профессионалов, обладающих именно тем набором знаний и технологий, которые необходимы заказчику. Разработчики, аналитики, тестировщики выделяются на длительный срок и работают в тесном сотрудничестве с ИТ-департаментом заказчика. Это дает специалистам возможность глубоко изучить особенности бизнеса, внутреннюю инфраструктуру компании и стать экспертом в конкретной области.

  \item поддержка и развитие существующих решений;
  От эффективности работ на этапе поддержки и сопровождения зависит непрерывность бизнес-процессов и сохранность корпоративной информации, необходимой для жизнедеятельности компаний. Однако, зачастую сопровождение информационных систем оказывается сложным и трудоемким процессом и требует много усилий со стороны ИТ-департаментов, отвлекая ресурсы от выполнения стратегически приоритетных задач. Это толкает компании прибегать к помощи сторонних компаний-разработчиков, предлагающих услуги по сопровождению информационных систем.
  \company{} обладает богатым опытом по выстраиванию процессов поддержки программного обеспечения 1,2,3 уровней у заказчиков из различных отраслей.
  Выделенная служба поддержки осуществляет:
  \begin{itemize}
    \item on-site и off-site сопровождение информационных систем заказчиков в соответствии с лучшими международными практиками;
    \item 1, 2, 3 уровни поддержки, SLA;
    \item поддержку информационных систем как собственной разработки, так и созданных сторонними подрядчиками;
    \item восстановление документации к информационным системам;
    \item фиксирование неполадок, устранение проблем в функционировании программного обеспечения;
    \item отслеживание и документирование дефектов;
    \item внесение изменений в документацию для расширения функциональности ПО;
    \item повышение эксплуатационных характеристик: быстродействия, надежности, отказоустойчивости и т.д.;
    \item сопровождение информационных систем заказчика в рамках выделенных центров разработки, согласно стандартам ITIL® и COBIT.
  \end{itemize}

  \item интеграция решений в существующую инфраструктуру;
  Традиционно информационная среда крупных компаний состоит из множества приложений, созданных в разное время разными разработчиками для решения конкретных бизнес-задач. Эти ИТ-системы, как правило, слабо связаны друг с другом на технологическом уровне, а данные в них часто не согласуются между собой. Со временем возникает необходимость в интеграции приложений и, что еще более важно, в надежных и опытных специалистах, которые смогут обеспечить связность всех бизнес-процессов и наладить непрерывное функционирование информационной среды.
  \company{} имеет обширный опыт проектирования, разработки и внедрения интеграционных решений на платформах лидеров этого сегмента рынка ПО: IBM WebSphere, Microsoft BizTalk, SAP NetWeaver, Oracle WebLogic, TIBCO ActiveEnterprise, технологии Open Source.
  Наши проекты в области интеграции приложений охватывают:
  \begin{itemize}
    \item разработку решений для интеграции корпоративных приложений (EAI), данных и бизнес-процессов как с внутренними, так и с внешними информационными системами компании-заказчика;
    \item проектирование архитектуры и внедрение интеграционного решения;
    \item разработку коннекторов к «унаследованным» приложениям различных производителей;
    \item комплексное тестирование интеграционных решений;
    \item разработку проектной документации;
    \item обследование и аудит сквозных бизнес-процессов и существующей ИТ-инфраструктуры;
    \item консультации по выбору и обоснованию платформы интеграционного решения;
    \item сопровождение и поддержку интеграционного решения;
    \item некоторые выполненные \company{} проекты по интеграции инфраструктуры крупных корпораций затрагивали одновременно многие десятки производственных площадок, сотни унаследованных приложений в нескольких странах разных континентов. Мы доказали, что чем сложнее поставленная задача, тем интереснее нам ее решать.
  \end{itemize}

  \item проведение тестирования и контроля качества;
  \company{} выполняет тестирование программного обеспечения как в рамках отдельных проектов, так и на различных этапах жизненного цикла ПО: проектирования, разработки, внедрения, промышленной эксплуатации, сопровождения и поддержки 2 и 3 уровней. \company{} проверяет функциональность, производительность, безопасность и другие характеристики самых различных приложений: разработанных \company{}, другими ИТ-компаниями или ИТ-подразделением заказчика.
  Для тестовых испытаний используются собственные стенды, облачные сервисы, виртуальные машины. Производственные процессы \company{} сертифицированы по стандарту информационной безопасности ISAE 3000 Type 2 (SAS70 Type II), а также соответствуют стандартам ISO 9001:2000 и CMMI level 4. Партнерство с IBM и другими технологическими лидерами помогает обеспечить необходимую тестовую среду для практически любых испытаний ПО:функционального, модульного, интеграционного тестирования; тестирования производительности, безопасности, баз данных,мобильных приложений.
  В большинстве проектов по разработке и тестированию программного обеспечения \company{} применяет автоматизированное тестирование с использованием как коммерческих инструментов: SilkTest, QuickTest Pro, WinRunner, TestComplete, Rational Functional Tester, Rational XDE Tester, Rational Robot, – так и open source-продуктов: Selenium, Watir, WatiN, HttpUnit, HtmlUnit, JUnit, NUnit, JaTeFW и т.п. Кроме этого, часто используется собственный инструментарий TAF (Test Automation Framework), который позволяет одновременно тестировать несколько решений, построенных на различных технологиях.
  Для успешной реализации крупных проектов практикуется создание выделенных центров тестирования – формируем команду и создаем инфраструктуру на базе одного из филиалов \company{} для тестирования ИТ-систем заказчика на постоянной основе. В компании также действует Центр компетенции по тестированию и контролю качества. Здесь аккумулируется экспертиза по тестированию горизонтальных и отраслевых ИТ-решений, происходит обмен компетенциями между профессионалами из офисов EPAM в различных странах мира.

  \item миграция приложений;
  Как известно, платформы и инструменты программирования достаточно быстро устаревают — значительно быстрее, чем бизнес-приложения, с помощью которых они создавались. Нередко возникает ситуация, когда применение новых технологий может вывести компанию на качественно новый уровень, но ключевые ИТ-системы компании не поддерживают необходимую платформу или технологию. Эффективно решить эту задачу позволяет миграция приложений (портирование программного обеспечения).
  \company{} имеет богатый опыт реализации проектов по миграции приложений, ручной и автоматизированной, для лидеров самых разных отраслей.
  В рамках проектов по портированию ПО \company{}:
  \begin{itemize}
    \item исследует портируемую информационную систему;
    \item изучает платформу, на которую осуществляется миграция;
    \item анализирует вероятные риски и затраты;
    \item переводит информационные системы заказчика на сервис-ориентированную архитектуру (SOA);
    \item осуществляет функциональное, нагрузочное, интеграционное тестирование;
    \item разрабатывает проектную и эксплуатационную документацию;
    \item проводит обучение бизнес-пользователей и администраторов системы;
    \item осуществляет поддержку и сопровождение программного обеспечения.
  \end{itemize}

  \item создание выделенных центров разработки для успешной реализации ИТ-проектов.
  В рамках выделенного центра EPAM предлагает своим клиентам услуги по разработке, тестированию и поддержке программного обеспечения.
  При организации выделенного центра \company{} формирует для клиента команду ИТ-специалистов и создает всю необходимую инфраструктуру. Работа центра организуется таким образом, что члены команды общаются с представителями заказчика напрямую и работают в общей территориально распределенной технической среде с заказчиком.
  Выбирая данную модель взаимодействия, \company{} обеспечивает максимальную прозрачность коммуникации между специалистами и клиентом. Для этого используется спектр технологий: общая виртуальная сеть (VPN), электронная почта, сервисы для мгновенного обмена сообщениями, традиционная и IP-телефония и др. Для контроля работ и управления центром применяется собственный инструментарий.
  При работе над проектом в формате выделенного центра \company{} стремится свести к минимуму бюрократизм управления единой командой центра.
\end{itemize}


ИООО~\company{} – это:
\begin{itemize}
  \item Инновационные проекты
  Масштаб EPAM позволяет нам работать с самыми крупными заказчиками над самыми сложными и интересными проектами, многие из которых инновационны по своей природе. Особое место среди проектов EPAM занимают разработки для мировых лидеров индустрии программного обеспечения (SAP, Hyperion и многих других), позволяющие занятым в них специалистам первыми приобрести уникальную технологическую экспертизу.
  \item Развитие
  Инвестиции в развитие сотрудников – один из принципов компании. EPAM предоставляет желающим возможность пройти внутренние и внешние тренинги, повысить уровень владения иностранным языком, расширить свою профессиональную компетенцию за счет работы над проектами в разных отраслях экономики и для заказчиков из разных стран.
  \item Финансовая стабильность
  EPAM предлагает сотрудникам широкий социальный пакет и конкурентную заработную плату, которая регулярно пересматривается и отражает достижения конкретного человека.
  \item Демократичная культура
  EPAM Systems – компания, совмещающая преимущества глобальности с вниманием к каждому сотруднику. Гибкий график работы (не 8 часов в день, а 40 часов в неделю) и отпусков, возможность работать удаленно при необходимости, прохождение практики, помощь коллег – все это служит выгодным фоном для поддержки сотрудников в достижении ими профессиональных целей.
\end{itemize}

В компании работает много молодых и зрелых специалистов.
Компания хорошо относится к своим сотрудникам, созданы условия для отдыха и развлечения сотрудников.
В офисе компании есть обеденная зона, снабженная столами, холодильником, микроволновой печью, чайником и бесплатным кофе-автоматом.Так же имеется комната для отдыха и развлечений, созданная специально для сотрудников, где можно сыграть в настольный футбол, настольный теннис или X-Box. В теплое время года компания часто организует активный отдых за городом для своих работников.

Прохождение преддипломной практики было в команде занимающейся разработкой и поддержкой платформы AppsNgen, предназначенной для разработки виджетов и распространение виджетов.
Результаты, полученные в ходе выполнения индивидуального задания, в данный момент используются компанией BB\&T для управления, приобретенными ими, виджетами.
