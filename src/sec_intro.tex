\sectioncentered*{Введение}
\addcontentsline{toc}{section}{Введение}
\label{sec:intro}

За последний век автомобили сыскали большую популярность среди людей как средство передвижения, на это много причин: и экономия времени, и комфорт, и удобство. Но вкупе с активной урбанизацией большое количество автомобилей становиться большой проблемой, ведь плотность населения в городе постоянно растет, особено удручает то, что редко при строительстве задумываются о проблемах с парковкой сейчас, не говоря о том что будет в будущем. Количества парковочных мест заложенных при строительстве современных, а со старыми зданиями дела обстоят несколько хуже, не хватает чтобы обеспечить потребностей. Повсюду в городе можно найти переполненные парковки. Даже правильным планированием парковочных мест на стадии строительстве оффиса задача свободной парковки не будет решена, потому как дефицит наблюдается в целом в городе. Поэтому верным шагом после правильного планирования будет ограничение доступа: пускать лишь тех для кого места и планировались. Типичным решением задачи контроля пропуска является установка шлагбаума, и существует несколько решений для его управленим. Целью моей дипломной работы является автоматическое управление шлагбаумом, основывающееся на распознавании государственных регистрационных знаках подъезжающего к стоянке автомобиля. 

