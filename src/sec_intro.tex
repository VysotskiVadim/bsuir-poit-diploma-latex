\sectioncentered*{Введение}
\addcontentsline{toc}{section}{Введение}
\label{sec:intro}

1908 год ознаменовался выпуском первого серийного автомобиля доступного среднему классу: Ford Model T~\cite{ford_t}. Начиная с этого времени количество автомобилей во всем мире непрерывно увеличивается, также растет потребность в парковочных местах.

Количества парковочных мест заложенных при строительстве современных, не говоря уже о старых зданиях, не хватает чтобы обеспечить потребностей. В результате получаем переполненные парковки, люди не могут найти места для своей машины и отправляются искать их возле соседних зданий. Зачастую работникам офисов не хватает места для своего автомобиля, не смотря на то что количество мест рядом с офисным зданием расчитанно исходя из количества работников. Напрашивается вывод: требуется ограничить доступ к парковке для посторонних лиц. Один из самых простых способов ограничения доступа это установка шлагбаума, разрешающего въезд и выезд только определенным автомобилям.

Появляется задача управления шлагбаумом. Что бы определить принадлежит ли автомобиль сотруднику компании машину нужно опознать. Ранее с этой задачей справлялся лишь человек, теперь же это можно автоматизировать.

Одним из самых точных идентификаторов автомобиля является его государственный регистрационный знак. Имея базу данных с автомобильными номерами сотрудников и возможность считать его пред въездом можно легко определить принадлежность автомобиля.

