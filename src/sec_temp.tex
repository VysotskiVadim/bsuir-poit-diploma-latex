\section{Напианный текст для не написанных глав}
\label{sec:temp}

\subsubsection{}
\label{sub:temp:image_processing:edge_detection}
Выделение границ

Выделение границ — выделение точек цифрового изображения, в которых резко изменяется яркость или есть другие виды неоднородностей~\cite{edge_detection}.
В идеальном случае, результатом выделения границ является набор связанных кривых, обозначающих границы объектов, граней и оттисков на поверхности, а также кривые, которые отображают изменения положения поверхностей. Таким образом, применение фильтра выделения границ к изображению может существенно уменьшить количество обрабатываемых данных, из-за того, что отфильтрованная часть изображения считается менее значимой, а наиболее важные структурные свойства изображения сохраняются. Однако не всегда возможно выделить границы в картинах реального мира средней сложности. Границы выделенные из таких изображений часто имеют такой недостаток как фрагментированость Но в задаче распознавания автомобильного номера для автомобильной стоянки камера будет находится недалеко от номера, так что границы номера будут отчетливо видны.

\subsection{Поиск номерного знака}
\label{sub:temp:search}
Задача поиска номерного знака имеет множество решений. Рассмотрим несколько из них.
\subsubsection{}
\label{sub:temp:search:hought}
Преобразование Хафа

Преобразование Хафа - алгоритм, численный метод, применяемый для извлечения элементов из изображения~\cite{hough_transform}. Используется в анализе изображений, цифровой обработке изображений и компьютерном зрении. Предназначен для поиска объектов, принадлежащих определённому классу фигур, с использованием процедуры голосования. Процедура голосования применяется к пространству параметров, из которого и получаются объекты определённого класса фигур по локальному максимуму в так называемом накопительном пространстве (accumulator space), которое строится при вычислении трансформации Хафа. Используя трансформацию Хафа мы будем искать границы номера, так что пред использованием требуется выделить границы на изображении. Трансформация Хафа поможет нам найти все прямые далее анализируя прямые мы можем найти границу номера. Преимущества алгоритма инвариантности номерного знака, и самый большой недостаток трансформации Хафа это медлительность. Считаю что для конкретной задачи поиска автомобильного номера лучше использовать алгоритм \segmentation.

\subsubsection{}
\label{sub:temp:search:segmentation}
\segmentation{}

Связная область это множество пикселей, у каждого из которых есть хотя бы один сосед, принадлежащий данному множеству. Для выделения связных областей есть есть 2 классических алгоритма:
\begin{itemize}
  \item Рекурсивный
  \item Последовательного сканирования
\end{itemize}
Рассмотрим алгоритм последовательного сканирования. При последовательном сканировании мы анализируем изображение слева на право, сверху вниз, сравнивая пиксели с левым и верхним соседом соседом. Если один из соседей принадлежит какой-либо области текущий пиксель также относим пиксель к этой области. В случае если соседи относятся к разным областям получается конфликт. Устанавливаем область пикселя в значение любого из соседей и идем дальше. После первого прохода идет второй проход: его цель решить проблему конфликтов. Если пиксели с разных сегментов находятся радом значит они принадлежат одному сегменту и их нужно объединить. 

В контексте задачи нахождения рамки с номером выделение областей лучше выполнять после выделения границ. Когда у нас есть связные области требуется найти прямоугольники среди них. Для этого лучше всего использовать алгоритм \minAreaRect{}. Поиск прямоугольников через связные области показывает куда лучшую производительность чем трансформация Хафа рассмотренная в главе \ref{sub:temp:search:hought}

\subsubsection{}
\label{sub:temp:search:selectrect}
Выбор прямоугольников.

Теперь когда есть прямоугольники из изображения требуется выбрать те что похожи на номер. Для этого воспользуемся государственным стандартом~\cite{stb_914_99}, в котором говорится соотношение сторон для номерных знаков.

