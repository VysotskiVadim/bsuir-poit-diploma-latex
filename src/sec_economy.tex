\newcommand{\byr}{Br}

\section{Технико-экономическое обоснование эффективности разработки программного средства}

% Begin Calculations

\FPeval{\totalProgramSize}{15680}
\FPeval{\totalProgramSizeCorrected}{8650}

\FPeval{\normativeManDays}{224}

\FPeval{\additionalComplexity}{0.12}
\FPeval{\complexityFactor}{clip(1 + \additionalComplexity)}

\FPeval{\stdModuleUsageFactor}{0.7}
\FPeval{\originalityFactor}{0.7}

\FPeval{\adjustedManDaysExact}{clip( \normativeManDays * \complexityFactor * \stdModuleUsageFactor * \originalityFactor )}
\FPround{\adjustedManDays}{\adjustedManDaysExact}{0}

\FPeval{\daysInYear}{365}
\FPeval{\redLettersDaysInYear}{9}
\FPeval{\weekendDaysInYear}{104}
\FPeval{\vocationDaysInYear}{21}
\FPeval{\workingDaysInYear}{ clip( \daysInYear - \redLettersDaysInYear - \weekendDaysInYear - \vocationDaysInYear ) }

\FPeval{\developmentTimeMonths}{3}
\FPeval{\developmentTimeYearsExact}{clip(\developmentTimeMonths / 12)}
\FPround{\developmentTimeYears}{\developmentTimeYearsExact}{2}
\FPeval{\requiredNumberOfProgrammersExact}{ clip( \adjustedManDays / (\developmentTimeYears * \workingDaysInYear) + 0.5 ) }

% тут должно получаться 2 ))
\FPtrunc{\requiredNumberOfProgrammers}{\requiredNumberOfProgrammersExact}{0}

\FPeval{\tariffRateFirst}{600000}
\FPeval{\tariffFactorFst}{3.04}
\FPeval{\tariffFactorSnd}{3.48}


\FPeval{\employmentFstExact}{clip( \adjustedManDays / \requiredNumberOfProgrammers )}
\FPtrunc{\employmentFst}{\employmentFstExact}{0}

\FPeval{\employmentSnd}{clip(\adjustedManDays - \employmentFst)}


\FPeval{\workingHoursInMonth}{160}
\FPeval{\salaryPerHourFstExact}{clip( \tariffRateFirst * \tariffFactorFst / \workingHoursInMonth )}
\FPeval{\salaryPerHourSndExact}{clip( \tariffRateFirst * \tariffFactorSnd / \workingHoursInMonth )}
\FPround{\salaryPerHourFst}{\salaryPerHourFstExact}{0}
\FPround{\salaryPerHourSnd}{\salaryPerHourSndExact}{0}

\FPeval{\bonusRate}{1.5}
\FPeval{\workingHoursInDay}{8}
\FPeval{\totalSalaryExact}{clip( \workingHoursInDay * \bonusRate * ( \salaryPerHourFst * \employmentFst + \salaryPerHourSnd * \employmentSnd ) )}
\FPround{\totalSalary}{\totalSalaryExact}{0}

\FPeval{\additionalSalaryNormative}{20}

\FPeval{\additionalSalaryExact}{clip( \totalSalary * \additionalSalaryNormative / 100 )}
\FPround{\additionalSalary}{\additionalSalaryExact}{0}

\FPeval{\socialNeedsNormative}{0.5}
\FPeval{\socialProtectionNormative}{34}
\FPeval{\socialProtectionFund}{ clip(\socialNeedsNormative + \socialProtectionNormative) }

\FPeval{\socialProtectionCostExact}{clip( (\totalSalary + \additionalSalary) * \socialProtectionFund / 100 )}
\FPround{\socialProtectionCost}{\socialProtectionCostExact}{0}

\FPeval{\taxWorkProtNormative}{4}
\FPeval{\taxWorkProtCostExact}{clip( (\totalSalary + \additionalSalary) * \taxWorkProtNormative / 100 )}
\FPround{\taxWorkProtCost}{\taxWorkProtCostExact}{0}
\FPeval{\taxWorkProtCost}{0} % это считать не нужно, зануляем чтобы не менять формулы

\FPeval{\stuffNormative}{3}
\FPeval{\stuffCostExact}{clip( \totalSalary * \stuffNormative / 100 )}
\FPeval{\stuffCost}{\stuffCostExact}

\FPeval{\timeToDebugCodeNormative}{15}
\FPeval{\reducingTimeToDebugFactor}{0.3}
\FPeval{\adjustedTimeToDebugCodeNormative}{ clip( \timeToDebugCodeNormative * \reducingTimeToDebugFactor ) }

\FPeval{\oneHourMachineTimeCost}{5000}

\FPeval{\machineTimeCostExact}{ clip( \oneHourMachineTimeCost * \totalProgramSizeCorrected / 100 * \adjustedTimeToDebugCodeNormative ) }
\FPround{\machineTimeCost}{\machineTimeCostExact}{0}

\FPeval{\businessTripNormative}{15}
\FPeval{\businessTripCostExact}{ clip( \totalSalary * \businessTripNormative / 100 ) }
\FPround{\businessTripCost}{\businessTripCostExact}{0}

\FPeval{\otherCostNormative}{20}
\FPeval{\otherCostExact}{clip( \totalSalary * \otherCostNormative / 100 )}
\FPround{\otherCost}{\otherCostExact}{0}

\FPeval{\overheadCostNormative}{100}
\FPeval{\overallCostExact}{clip( \totalSalary * \overheadCostNormative / 100 )}
\FPround{\overheadCost}{\overallCostExact}{0}

\FPeval{\overallCost}{clip( \totalSalary + \additionalSalary + \socialProtectionCost + \taxWorkProtCost + \stuffCost + \machineTimeCost + \businessTripCost + \otherCost + \overheadCost ) }

\FPeval{\supportNormative}{30}
\FPeval{\softwareSupportCostExact}{clip( \overallCost * \supportNormative / 100 )}
\FPround{\softwareSupportCost}{\softwareSupportCostExact}{0}


\FPeval{\baseCost}{ clip( \overallCost + \softwareSupportCost ) }

\FPeval{\profitability}{35}
\FPeval{\incomeExact}{clip( \baseCost / 100 * \profitability )}
\FPround{\income}{\incomeExact}{0}

\FPeval{\estimatedPrice}{clip( \income + \baseCost )}

\FPeval{\localRepubTaxNormative}{3.9}
\FPeval{\localRepubTaxExact}{clip( \estimatedPrice * \localRepubTaxNormative / (100 - \localRepubTaxNormative) )}
\FPround{\localRepubTax}{\localRepubTaxExact}{0}
\FPeval{\localRepubTax}{0}

\FPeval{\ndsNormative}{20}
\FPeval{\ndsExact}{clip( (\estimatedPrice + \localRepubTax) / 100 * \ndsNormative )}
\FPround{\nds}{\ndsExact}{0}


\FPeval{\sellingPrice}{clip( \estimatedPrice + \localRepubTax + \nds )}

\FPeval{\taxForIncome}{18}
\FPeval{\incomeWithTaxes}{clip(\income * (1 - \taxForIncome / 100))}
\FPround\incomeWithTaxes{\incomeWithTaxes}{0}

% End Calculations

\subsection{Характеристика программного продукта }

Предлагаемый для внедрения программный продукт является автоматизированной системой распознавания номеров автомобилей. Данная система осуществляет распознавание номерных знаков автомобилей, регистрирует и принимает решение об открытии шлагбаума. 
Разработка продукта осуществлена в IT-компании «Эпам системз» для собственных нужд. В результате внедрения предлагаемой системы удается достичь:
\begin{itemize}
  \item автоматический контроль пропуска автомобилей;
  \item предоставление данных о заехавших автомобилях; 
\end{itemize}
Разработка данного проекта связана с затратами трудовых и финансовых ресурсов. В связи с этим создание и реализация данного проекта нуждается в соответствующем технико-экономическом обосновании.

Целью технико-экономического обоснования является расчет и оценка следующих показателей:
\begin{itemize}
  \item смета затрат и отпускная цена ПО;
  \item прибыль от реализации ПО;
  \item рентабельность инвестиций в разработку ПО.
\end{itemize}
В результате разработки данного программного продукта снизится трудоемкость контроля доступа к корпоративной парковке.


\subsection{Определение объема и трудоемкости ПО}

\subsubsection{Объем ПО}

Базой для расчета плановой сметы затрат на разработку ПО является объем ПО.   

\subsubsection{Общий объем ($ V_{o} $)  }
программного продукта определяется исходя из количества и объема функций, реализуемых программой:

\begin{equation}
  \label{eq:econ:total_program_size}
  V_{o} = \sum_{i = 1}^{n} V_{i} \text{\,,}
\end{equation}
\begin{explanation}
где & $ V_{i} $ & объём отдельной функции ПО, LoC; \\
    & $ n $ & общее число функций.
\end{explanation}

\subsubsection{Единицы измерения объема ПО}
Оценивание объема программного
продукта связано с выбором наиболее подходящей единицы измерения размера продукта. В данном обосновании будет использоваться количество строк исходного кода (LinеsОfСоdе, LОС). 

Рассчитывается уточненный объем ПО ($ V_{\text{у}}$):

\begin{equation}
  \label{eq:econ:total_program_size_corrected}
  V_{\text{у}} = \sum_{i = 1}^{n} V_{i}^{\text{у}} \text{\,,}
\end{equation}
\begin{explanation}
где & $ V_{i}^{\text{y}} $ & уточненный объём отдельной функции ПО, LoC; \\
    & $ n $ & общее число функций.
\end{explanation}

Среда разработки ПО - \dotnet{}, ПО функционального назначения. $ V_{i}$ = 2650 LOC.


\begin{table}[ht]
\caption{Перечень и объём функций программного модуля}
\label{table:econ:function_sizes}
\centering
  \begin{tabular}{| >{\centering}m{0.12\textwidth} 
                  | >{\raggedright}m{0.40\textwidth} 
                  | >{\centering}m{0.18\textwidth} 
                  | >{\centering\arraybackslash}m{0.18\textwidth}|}

  \hline
         \multirow{2}{0.12\textwidth}[-0.5em]{\centering \No{} функции}
       & \multirow{2}{0.40\textwidth}[-0.55em]{\centering Наименование (содержание)} 
       & \multicolumn{2}{c|}{\centering Объём функции, LoC} \tabularnewline
  
  \cline{3-4} & 
       & { по каталогу ($ V_{i} $) }
       & { уточненный ($ V_{i}^{\text{у}} $) } \tabularnewline
  
  \hline 
  101 & Организация ввода информации & \num{50} & \num{50} \tabularnewline
  
  \hline
  305 & Организация конфигурационных файлов & \num{750} & \num{750} \tabularnewline

  \hline
  210 & Загрузка данных & \num{550} & \num{550} \tabularnewline

  \hline
  501 & Обеспечение интерфейса между компонентами & \num{730} & \num{730} \tabularnewline

  \hline
  503 & Работа с IP камерой & \num{550} & \num{550} \tabularnewline

  \hline
  705 & Формирование ввода и вывода на внешние носители & \num{390} & \num{390} \tabularnewline

  \hline
  707 & Графический вывод результатов & \num{300} & \num{300} \tabularnewline

  \hline

  % Уточенная оценка вычислялась с помощью R: (+ручной фикс)
  % set.seed(35)
  % locs <- c(100, 520, 2700, 520, 750, 1100, 430, 730, 460, 8370)
  % locs.which.corrected <- rbinom(length(locs), 1, 0.4)
  % locs.corrections <- rnorm(length(locs), mean = -0.25, sd=0.3)
  % locs.correction.factor <- 1 + locs.which.corrected * locs.corrections
  % locs.corrected <- signif(locs * locs.correction.factor, digits = 2)
  % locs.corrected
  % sum(locs)
  % sum(locs.corrected)

  Итог & & \num{3050} & \num{3050} \tabularnewline

  \hline

  \end{tabular}
\end{table}

\subsubsection{Трудоемкость разработки ПО}

По уточненному объему ПО и нормативам затрат труда в расчете на единицу объема определяются нормативная и общая трудоемкость разработки ПО.

\subsubsection{Нормативная трудоемкость разработки ПO}

На основании принятого к расчету объема (Vy) и категории сложности определяется нормативная трудоемкость ПО (Tн), которая уточняется с учетом сложности и новизны проекта и степени использования стандартных модулей при разработке.

\subsubsection{Общая трудоемкость разработки ПО}

Нормативная трудоемкость (Tн) служит основой для определения общей трудоемкости (Tо), расчет которой осуществляется различными способами в зависимости от размера проекта.

\subsubsection{Общая трудоемкость} 
Общая трудоемкость проектов рассчитывается по формуле:

\begin{equation}
  \label{eq:econ:effort_common}
  \text{Т}_\text{о} = \text{Т}_\text{н} \cdot 
                      \text{К}_\text{с} \cdot 
                      \text{К}_\text{т} \cdot 
                      \text{К}_\text{н} \text{\,,}
\end{equation}
\begin{explanation}
где & $ \text{К}_\text{с} $ & коэффициент, учитывающий сложность ПО; \\
    & $ \text{К}_\text{т} $ & поправочный коэффициент, учитывающий степень использования при разработке стандартных модулей; \\
    & $ \text{К}_\text{н} $ & коэффициент, учитывающий степень новизны ПО.
\end{explanation}

Наличие интерактивного доступа и обеспечения хранения, ведения и поиска данных в сложных структурах позволяет применить к объему ПО коэффициент Кс, который определяется по формуле:

\begin{equation}
\label{eq:econ:complexity_coeff}
  \text{К}_{\text{с}} = 1 + \sum_{i = 1}^n \text{К}_{i} \text{\,,}
\end{equation}
\begin{explanation}
где & $ \text{К}_{i} $ & коэффициент, соответствующий степени повышения сложности ПО за счет конкретной характеристики; \\
    & $ n $ & количество учитываемых характеристик.
\end{explanation}


\begin{equation}
\label{eq:econ:complexity_coeff_calc}
  \text{К}_{\text{с}} = \num{1} + \num{0,07} + \num{0,06} = \num{1,13} \text{\,.}
\end{equation}

С учетом дополнительного коэффициента сложности Кс рассчитывается общая трудоемкость ПС по формуле:
\begin{equation}
  \label{eq:econ:effort_common_calc}
  \text{Т}_\text{о} = \num{745,8} \approx \num{746}{\text{чел.}/\text{дн.}}
\end{equation}


\subsubsection{Численность исполнителей и срок разработки ПО}

На основе общей трудоемкости и требуемых сроков реализации проекта вычисляется плановое количество исполнителей. При этом могут решаться следующие задачи:

\begin{itemize}
  \item расчет числа исполнителей при заданных сроках разработки;
  \item определение сроков разработки проекта при заданной численности исполнителей.
\end{itemize}

\subsubsection{Численность исполнителей}

Численность исполнителей проекта рассчитывается по формуле:

\begin{equation}
  \label{eq:econ:num_of_programmers}
  \text{Ч}_\text{р} = \frac{\text{Т}_\text{о}}{\text{Т}_\text{р} \cdot \text{Ф}_\text{эф}} \text{\,,}
\end{equation}
\begin{explanation}
где & $ \text{Т}_\text{о} $ & общая трудоемкость разработки проекта, $ \text{чел.}/\text{дн.} $; \\
    & $ \text{Ф}_\text{эф} $ & эффективный фонд времени работы одного работника в течение года, дн.; \\
    & $ \text{Т}_\text{р} $ & срок разработки проекта, лет.
\end{explanation}

\subsubsection{Срок разработки проекта}

Срок разработки проекта рассчитывается по формуле:

\begin{equation}
  \label{eq:econ:num_of_programmers_c}
  \text{Т}_\text{р} = \frac{\text{Т}_\text{о}}{\text{Ч}_\text{р} \cdot \text{Ф}_\text{эф}} \text{\,,}
\end{equation}
\begin{explanation}
где & $ \text{Т}_\text{о} $ & общая трудоемкость разработки проекта, $ \text{чел.}/\text{дн.} $; \\
    & $ \text{Ф}_\text{эф} $ & эффективный фонд времени работы одного работника в течение года, дн.; \\
    & $ \text{Т}_\text{р} $ & срок разработки проекта, лет.
\end{explanation}


\subsubsection{Эффективный фонд времени работы}

Эффективный фонд времени работы одного работника рассчитывается по формуле:
\begin{equation}
  \label{eq:econ:effective_time_per_programmer}
  \text{Ф}_\text{эф} = 
    \text{Д}_\text{г} -
    \text{Д}_\text{п} -
    \text{Д}_\text{в} -
    \text{Д}_\text{о} \text{\,,}
\end{equation}
\begin{explanation}
где & $ \text{Д}_\text{г} $ & количество дней в году, дн.; \\
    & $ \text{Д}_\text{п} $ & количество праздничных дней в году, не совпадающих с выходными днями, дн.; \\
    & $ \text{Д}_\text{в} $ & количество выходных дней в году, дн.; \\
    & $ \text{Д}_\text{п} $ & количество дней отпуска, дн.
\end{explanation}

Трудоемкость <<Технорабочего проекта>> определяется по формуле:

\begin{equation}
  \label{eq:econ:effective_time_per_programmer_df}
  \text{Т}_\text{трп} = 
    \num{0,85} \cdot 
    \text{Т}_\text{тп} +
    \num{1} -
    \text{Т}_\text{рп} \text{\,.}
\end{equation}

Она составляет: 

\begin{equation}
  \label{eq:econ:effective_time_per_programmer_dfr}
  \text{Т}_\text{трп} = 
    \num{0,85} \cdot 
    \num{36,54} +
    \num{1} -
    \num{222,92} =
    \num{254} {\text{чел.}/\text{дн}} \text{\,.}
\end{equation}

На основе уточненной трудоемкости разработки ПС и установленного периода разработки, общая численность разработчиков равна:

\begin{equation}
  \text{Ф}_\text{эф} = \num{\daysInYear} - \num{5} - \num{106} - \num{31} = \SI{223}{\text{дня в год}}
\end{equation}

\subsection{Расчет затрат и отпускной цены программного средства}

1) Основная заработная плата исполнителей проекта определяется по формуле:
\begin{equation}
  \label{eq:econ:total_salary}
  \text{З}_{\text{о}} = \sum^{n}_{i = 1} 
                        \text{Т}_{\text{ч}}^{i} \cdot
                        \text{Т}_{\text{ч}} \cdot
                        \text{Ф}_{\text{п}} \cdot
                        \text{К}
                          \text{\,,}
\end{equation}
\begin{explanation}
где & $ \text{Т}_{\text{ч}}^{i} $ & часовая тарифная ставка \mbox{$ i $-го} исполнителя, \byr$/$час; \\
    & $ \text{Т}_{\text{ч}} $ & количество часов работы в день, час; \\
    & $ \text{Ф}_{\text{п}} $ & плановый фонд рабочего времени \mbox{$ i $-го} исполнителя, дн.; \\
    & $ \text{К} $ & коэффициент премирования.
\end{explanation}

В настоящий момент тарифная ставка 1-го разряда на предприятии составляет 700 тыс. руб. 

Расчет основной заработной платы представлен в таблице~\ref{table:econ:programmers_zp}.

\begin{table}[ht]
  \caption{Расчет основной заработной платы}
  \label{table:econ:programmers_zp}
  \begin{tabular}{| >{\raggedright}p{0.17\textwidth} 
                  | >{\raggedright}p{0.08\textwidth} 
                  | >{\raggedright}p{0.13\textwidth}
                  | >{\raggedright}p{0.12\textwidth}
                  | >{\raggedright}p{0.10\textwidth}
                  | >{\raggedright}p{0.10\textwidth} 
                  | >{\raggedright\arraybackslash}p{0.12\textwidth}|}
   \hline
   Исполнители & Разряд & Тарифный коэффициент & Месячная тарифная ставка тыс.руб. & Часовая тарифная ставка тыс.руб. & Плано-вый фонд рабочего времени, дн. & Основная заработная плата, тыс. руб.\\
   \hline
   Программист \Rmnum{2}-категории & $ \num{10} $ & $ \num{2,48} $ & $ \num{1736} $ & $ \num{28603} $& $ \num{22} $  & $ \num{5006} $\\
   
   \hline
  \end{tabular}
\end{table}

2) Дополнительная заработная плата исполнителей проекта определяется по формуле:

\begin{equation}
  \label{eq:econ:additional_salary}
  \text{З}_{\text{д}} = 
    \frac {\text{З}_{\text{о}} \cdot \text{Н}_{\text{д}}} 
          {100\%} \text{\,,}
\end{equation}
\begin{explanation}
  где & $ \text{Н}_{\text{д}} $ & норматив дополнительной заработной платы, $ \% $.
\end{explanation}

Дополнительная заработная плата составит:

\begin{equation}
  \label{eq:econ:additional_salary_calc}
  \text{З}_{\text{д}} = 
    \frac{\num{5006} \cdot 20\%}
         {100\%} \approx \SI{1 001 }{\text{тыс.руб}} \text{\,.}
\end{equation}

3) Отчисления в фонд социальной защиты населения и на обязательное страхование (ЗС) определяются в соответствии с действующими законодательными актами по формуле:

\begin{equation}
  \label{eq:econ:soc_prot}
  \text{З}_{\text{сз}} = 
    \frac{(\text{З}_{\text{о}} + \text{З}_{\text{д}}) \cdot \text{Н}_{\text{сз}}}
         {\num{100\%}} \text{\,.}
\end{equation}

Подставив вычисленные ранее значения в формулу получаем:
\begin{equation}
  \label{eq:econ:soc_prot_calc}
  \text{З}_{\text{сз}} =
    \frac{ (\num{5006} + \num{1001}) \cdot \num{34,6\%} }
         { \num{100\%} }
    \approx \SI{2078}{\text{тыс.руб}} \text{\,.}
\end{equation}

4) Расходы по статье <<Машинное время>> (РМ) включают оплату машинного времени, необходимого для разработки и отладки ПС, и определяются по формуле:

\begin{equation}
  \label{eq:econ:machine_time}
  \text{Р}_{\text{м}} =
    \text{Ц}_{\text{м}} \cdot 
    \text{Т}_{\text{ч}} \cdot 
    \text{С}_{\text{р}}
    \text{\,,}
\end{equation}
\begin{explanation}
  где & $ \text{Ц}_{\text{м}} $ & цена одного часа машинно-часа, бел.руб.; \\
      & $ \text{Т}_{\text{ч}} $ & количество часов работы в день; \\
      & $ \text{С}_{\text{р}} $ & длительность проекта.
\end{explanation}

Стоимость машино-часа на предприятии составляет 6 тыс. руб.. Разработка проекта займет 90 дней. Определим затраты по статье <<Машинное время>>:

\begin{equation}
  \label{eq:econ:machine_time}
  \text{Р}_{\text{м}} =
    \num{6} \cdot 
    \num{4} \cdot 
    \num{88} =
    \SI{2112}{\text{тыс.руб}} \text{\,.}
\end{equation}


5) Затраты по статье <<Накладные расходы>> (РН), определяются по формуле:

\begin{equation}
  \label{eq:econ:overhead_cost}
  \text{Р}_{\text{н}} =
    \frac{ \text{З}_{\text{о}} \cdot \text{Н}_{\text{рн}} }
         { \num{100\%} } \text{\,,}
\end{equation}
\begin{explanation}
  где & $ \text{Н}_{\text{рн}} $ & норматив накладных расходов в организации,~$ \num{50\%} $.
\end{explanation}

Накладные расходы составят:

\begin{equation}
  \label{eq:econ:overhead_cost_calc}
  \text{Р}_{\text{н}} =
   \text{Р}_{\text{н}} =
    \num{5006} \cdot 
    \num{0.5} = 
    \SI{2503}{\text{тыс.руб}} \text{\,.}
\end{equation}

Общая сумма расходов по всем статьям сметы (Сп) на ПО рассчитывается по формуле:

\begin{equation}
  \label{eq:econ:overall_cost}
  \text{С}_{\text{р}} =
    \text{З}_{\text{о}} +
    \text{З}_{\text{д}} +
    \text{З}_{\text{сз}} +
    \text{Р}_{\text{м}} +
    \text{Р}_{\text{н}}\text{\,.}
\end{equation}

Подставляя ранее вычисленные значения в формулу получаем:
\begin{equation}
  \label{eq:econ:overall_cost_calc}
  \text{С}_{\text{р}} =
    \num{5006} \cdot 
    \num{1001} \cdot 
    \num{2078} \cdot 
    \num{2112} \cdot 
    \num{2503} = \SI{12699,8}{\text{тыс. руб}} \text{\,.}
\end{equation}

Расходы на сопровождение и адаптацию, которые несет производитель ПО, вычисляются по нормативу от суммы расходов по смете и рассчитываются по формуле:
\begin{equation}
  \label{eq:econ:software_support}
  \text{Р}_{\text{са}} = 
    \frac { \text{С}_{\text{р}} \cdot \text{Н}_{\text{рса}} }
          { \num{100\%} } \text{\,,}
\end{equation}
\begin{explanation}
  где & $ \text{Н}_{\text{рса}} $ & норматив расходов на сопровождение и адаптацию ПО,~$ \num{20\%} $.
\end{explanation}


\begin{equation}
  \label{eq:econ:software_support_calc}
  \text{Р}_{\text{са}} = 
    \frac { \num{12699,8} \cdot \num{20\%} }
          { \num{100\%} } \approx \SI{2540}{\text{тыс. руб}} \text{\,.}
\end{equation}

Общая сумма расходов на разработку (с затратами на сопровождение и адаптацию) как полная себестоимость ПС (СП) определяется по формуле:

\begin{equation}
  \label{eq:econ:base_cost}
  \text{С}_{\text{п}} = \text{С}_{\text{р}} + \text{Р}_{\text{са}} \text{\,.}
\end{equation}

Подставляя известные значения в формулу получаем:
\begin{equation}
  \label{eq:econ:base_cost_calc}
  \text{С}_{\text{п}} = \num{12700} + \num{2540} = \SI{15240}{\text{тыс. руб}} \text{\,.}
\end{equation}

Прибыль ПС рассчитывается по формуле:

\begin{equation}
  \label{eq:econ:income}
  \text{П}_{\text{с}} = 
    \frac { \text{С}_{\text{п}} \cdot \text{У}_{\text{р}} }
          { \num{100\%} } \text{\,,}
\end{equation}
\begin{explanation}
  где & $ \text{П}_{\text{с}} $ & прибыль от реализации ПО заказчику, тыс.руб.; \\
      & $ \text{У}_{\text{р}} $ & уровень рентабельности ПО,~$ \num{25\%} $; \\
      & $ \text{С}_{\text{п}} $ & себестоимость ПС (руб.).
\end{explanation}

Подставив известные данные в формулу получаем:
\begin{equation}
  \label{eq:econ:income_calc}
  \text{П}_{\text{с}} = 
    \frac { \num{15240} \cdot \num{25\%} }
          { \num{100\%} } 
    \approx \SI{3810}{\text{тыс.руб}} \text{\,.}
\end{equation}

Прогнозируемая отпускная цена ПС:
\begin{equation}
  \label{eq:econ:estimated_price}
  \text{Ц}_{\text{п}} = \text{С}_{\text{п}} + \text{П}_{\text{с}}  \text{\,.}
\end{equation}

Подставив данные в формулу получаем:
\begin{equation}
  \label{eq:econ:estimated_price_calc}
  \text{Ц}_{\text{п}} = \num{15240}  + \num{3810} = \SI{19050}{\text{тыс.руб}} \text{\,.}
\end{equation}

\subsection{Расчет стоимостной оценки результата}

Результатом (Р) в сфере использования программного продукта является прирост чистой прибыли и амортизационных отчислений.

\subsubsection{Расчет прироста чистой прибыли}

1) Экономия затрат на заработную плату при использовании ПС в расчете на объем выполняемых работ определяется по формуле:

\begin{equation}
  \label{eq:econ:incomex}
  \text{Э}_{\text{з}} = 
    \text{К}_{\text{пр}} \cdot 
    (\text{Т}_{\text{с}} \cdot  \text{ТС} - \text{Т}_{\text{н}} \cdot  \text{ТН}) \cdot 
    \text{N}_{\text{п}} \cdot 
    (\num{1} + \text{Н}_{\text{д}}/\num{100}) \cdot 
    (\num{1} + \text{Н}_{\text{но}}/\num{100})\text{\,,}
\end{equation}
\begin{explanation}
  где & $ \text{Т}_{\text{с}} $, $ \text{Т}_{\text{н}} $ & трудоемкость выполнения работы до и после внедрения программного продукта, нормо-час; \\
      & $ \text{ТС} $, $ \text{ТН} $ & часовая тарифная ставка, соответствующая разряду выполняемых работ до и после внедрения программного продукта, тыс. руб./ч.; \\
      & $ \text{К}_{\text{пр}} $ & коэффициент премий; \\
      & $ \text{Н}_{\text{д}} $ & норматив дополнительной заработной платы; \\
      & $ \text{Н}_{\text{но}} $ & ставка отчислений от заработной платы, включаемых в себестоимость. \\
\end{explanation}

До внедрения программного продукта трудоемкость контроля пропуска составляла 3 человека-часа, после внедрения программы – 1 человека-часа. 
Экономия на заработной и начисления на заработную плату составит:

\begin{equation}
  \label{eq:econ:estimated_price_calcdd}
  \text{Э}_{\text{з}} = \num{1,35} \cdot  (\num{1,3} \cdot  \num{8002} - \num{0,5} \cdot  \num{8002}) \cdot  \num{300} \cdot  \num{1,2} \cdot  \num{1,346} = \SI{4188}{\text{тыс.руб}} \text{\,.}
\end{equation}

Прирост чистой прибыли (ΔПч) определяется по формуле:

\begin{equation}
  \label{eq:econ:incomex}
  \text{ΔП}_{\text{ч}} = 
    \text{С}_{\text{о}} -
    ((\text{С}_{\text{о}} \cdot  \text{Н}_{\text{п}})/100)\text{\,,}
\end{equation}
\begin{explanation}
  где & $ \text{Н}_{\text{п}} $ & ставка налога на прибыль; \\
\end{explanation}

Таким образом, прирост чистой прибыли составит:

\begin{equation}
  \label{eq:econ:estimated_price_calcdd}
  \text{ΔП}_{\text{ч}} = \num{4990} - \num{4188} \cdot  \num{18} / \num{100} = \SI{4237}{\text{тыс.руб}} \text{\,,}
\end{equation}
  
\subsubsection{Расчет прироста амортизационных отчислений}

Расчет амортизационных отчислений осуществляется по формуле:

\begin{equation}
  \label{eq:econ:incomex}
  \text{А} = 
    \text{Н}_{\text{а}} \cdot 
    ((\text{З}/100)\text{\,.}
\end{equation}
\begin{explanation}
  где & $ \text{З} $ & затраты на разработку программы, тыс. руб.; \\
      & $ \text{Н}_{\text{а}} $ & норма амортизации программного продукта. \\
\end{explanation}

Таким образом, получим:

\begin{equation}
  \label{eq:econ:estimated_price_calcdd}
  \text{А} = \num{19050} \cdot  \num{0,2} = \SI{3810}{\text{тыс.руб}} \text{\,}
\end{equation}

\subsection{Расчет показателей эффективности использования программного продукта}

Для расчета показателей экономической эффективности использования программного продукта необходимо полученные суммы результата (прироста чистой прибыли) и затрат (капитальных вложений) по годам приводят к единому времени – расчетному году (за расчетный год принят 2015 год) путем умножения результатов и затрат за каждый год на коэффициент привидения ($ \text{ALFA}_{\text{t}} $), который рассчитывается по формуле:


\begin{equation}
  \label{eq:econ:incomex}
  \text{ALFA}_{\text{t}} = 
    (\num{1} + \text{E}_{\text{н}})^{\text{t}_{\text{p}} - \text{t}}\text{\,,}
\end{equation}
\begin{explanation}
  где & $ \text{E}_{\text{н}} $ & норматив привидения разновременных затрат и результатов; \\
      & $ \text{t}_{\text{p}} $ & расчетный год. \\
\end{explanation}

Результаты расчета показателей эффективности приведены в таблице 7.3.

Проект планируется внедрить в организации во второй половине 2016 года, поэтому в 2016 году организация может получить половину прибыли.

\includepdf[pages={-}, pagecommand={}]{economic_table.pdf}

\FPeval{\firstYearVariableCosts}{17200}
\FPeval{\secondYearVariableCosts}{18300}
\FPeval{\thirdYearVariableCosts}{15800}

\FPeval{\firstYearConstCosts}{12000}
\FPeval{\secondYearConstCosts}{11100}
\FPeval{\thirdYearConstCosts}{9400}

\FPeval{\firstYearCostOfProduction}{clip(\firstYearVariableCosts + \firstYearConstCosts)}
\FPeval{\secondYearCostOfProduction}{clip(\secondYearVariableCosts + \secondYearConstCosts)}
\FPeval{\thirdYearCostOfProduction}{clip(\thirdYearVariableCosts + \thirdYearConstCosts)}

\FPeval{\firstYearGrossProfit}{clip(\revenueInFirstYear - \firstYearCostOfProduction)}
\FPeval{\secondYearGrossProfit}{clip(\revenueInSecondYear - \secondYearCostOfProduction)}
\FPeval{\thirdYearGrossProfit}{clip(\revenueInThirdYear - \thirdYearCostOfProduction)}

\FPeval{\firstYearPropertyTax}{300}
\FPeval{\secondYearPropertyTax}{300}
\FPeval{\thirdYearPropertyTax}{300}

\FPeval{\firstYearTaxableIncome}{clip(\firstYearGrossProfit - \firstYearPropertyTax)}
\FPeval{\secondYearTaxableIncome}{clip(\secondYearGrossProfit - \secondYearPropertyTax)}
\FPeval{\thirdYearTaxableIncome}{clip(\thirdYearGrossProfit - \thirdYearPropertyTax)}

\FPeval{\incomeTax}{0.12}
\FPeval{\fyIncomeTax}{clip(\firstYearTaxableIncome * \incomeTax)}
\FPeval{\syIncomeTax}{clip(\secondYearTaxableIncome * \incomeTax)}
\FPeval{\tyIncomeTax}{clip(\thirdYearTaxableIncome * \incomeTax)}

\FPeval{\fyNetIncome}{clip(\firstYearTaxableIncome - \fyIncomeTax)}
\FPeval{\syNetIncome}{clip(\secondYearTaxableIncome - \syIncomeTax)}
\FPeval{\tyNetIncome}{clip(\thirdYearTaxableIncome - \tyIncomeTax)}

\FPeval{\capitalInvestments}{clip(\baseCost / 1000)}
\FPround\capitalInvestments{\capitalInvestments}{0}

\FPeval{\discountRate}{0.295}

\FPeval{\zyDiscountFactor}{1}
\FPeval{\fyDiscountFactor}{clip(\zyDiscountFactor / (1 + \discountRate))}
\FPeval{\syDiscountFactor}{clip(\fyDiscountFactor / (1 + \discountRate))}
\FPeval{\tyDiscountFactor}{clip(\syDiscountFactor / (1 + \discountRate))}


\FPeval{\fyCashFlowOutcome}{clip(\firstYearCostOfProduction + \firstYearPropertyTax + \fyIncomeTax)}
\FPeval{\syCashFlowOutcome}{clip(\secondYearCostOfProduction + \secondYearPropertyTax + \syIncomeTax)}
\FPeval{\tyCashFlowOutcome}{clip(\thirdYearCostOfProduction + \thirdYearPropertyTax + \tyIncomeTax)}

\FPeval{\fyCashFlowOutcomeDiscounted}{clip(\fyCashFlowOutcome * \fyDiscountFactor)}
\FPeval{\syCashFlowOutcomeDiscounted}{clip(\syCashFlowOutcome * \syDiscountFactor)}
\FPeval{\tyCashFlowOutcomeDiscounted}{clip(\tyCashFlowOutcome * \tyDiscountFactor)}

\FPeval{\fyCashFlowIncomeDiscounted}{clip(\firstYearCashFlowIncome * \fyDiscountFactor)}
\FPeval{\syCashFlowIncomeDiscounted}{clip(\secondYearCashFlowIncome * \syDiscountFactor)}
\FPeval{\tyCashFlowIncomeDiscounted}{clip(\thirdYearCashFlowIncome * \tyDiscountFactor)}

\FPround\fyDiscountFactor{\fyDiscountFactor}{3}
\FPround\syDiscountFactor{\syDiscountFactor}{3}
\FPround\tyDiscountFactor{\tyDiscountFactor}{3}

\FPround\fyCashFlowIncomeDiscounted{\fyCashFlowIncomeDiscounted}{0}
\FPround\syCashFlowIncomeDiscounted{\syCashFlowIncomeDiscounted}{0}
\FPround\tyCashFlowIncomeDiscounted{\tyCashFlowIncomeDiscounted}{0}

\FPround\fyCashFlowOutcomeDiscounted{\fyCashFlowOutcomeDiscounted}{0}
\FPround\syCashFlowOutcomeDiscounted{\syCashFlowOutcomeDiscounted}{0}
\FPround\tyCashFlowOutcomeDiscounted{\tyCashFlowOutcomeDiscounted}{0}

\FPround\zyNetIncomeAcc{-\capitalInvestments}{0}
\FPeval{\fyNetIncomeAcc}{clip(\zyNetIncomeAcc + (\fyCashFlowIncomeDiscounted - \fyCashFlowOutcomeDiscounted))}
\FPeval{\syNetIncomeAcc}{clip(\fyNetIncomeAcc + (\syCashFlowIncomeDiscounted - \syCashFlowOutcomeDiscounted))}
\FPeval{\tyNetIncomeAcc}{clip(\syNetIncomeAcc + (\tyCashFlowIncomeDiscounted - \tyCashFlowOutcomeDiscounted))}
